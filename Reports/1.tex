\documentclass[12pt]{article}

\usepackage{amsmath,amsthm,amssymb,amsfonts,setspace}
\usepackage{fontspec}
\usepackage{polyglossia}
\pagestyle{empty}
\setdefaultlanguage[babelshorthands=true]{russian}
\newfontfamily{\cyrillicfont}{Times New Roman}

\textwidth 7.0 truein
\oddsidemargin -0.25in   %left-hand edge
\evensidemargin -0.5 truein  %right-hand edge
\topmargin -0.85in      %top of paper to top of head, pulls whole unit
\textheight 9.5in

%%%% In most cases you won't need to edit anything above this line %%%%

\begin{document}
\hfill \textbf{Алексей Кацман}

\hfill \textbf{Университет ИТМО}

\hfill \textbf{Lip Reading Coursework}

\hfill \textbf{15 декабря 2015}

\bigskip

В общем-то немного о том, что я сделал.
Я написал некую базовую модель, с помощью которой можно анализировать эффективность работы написанного алгоритма. Для этого нужны видео и файл с субтитрами к нему. В качестве первого видео я решил взять некое не очень длинное видео с онлайн-лекции на coursera.org длительностью 5 минут и субтитры к нему. 

\bigskip

Собственно сам алгоритм, по которому определялось, говорит человек или нет, использовалась разность вертикальных точек и горизонтальных точек, а именно, расширение по вертикали до определенного размера либо сужение по горизонтали до определенного размера.

\bigskip

Теперь немного о том, как анализировал эффективность алгоритма. Я начинал прогонять свой алгоритм и выдавать текущее соотношение эффективности отрезками примерно по 5 секунд. На каждом из таких отрезков я смотрел, когда человек говорит на самом деле (по субтитрам) и когда человек говорит согласно написанному алгоритму. Далее я брал объединение всех отрезков с 2 множеств и находил их суммарную длительность. После чего оценивал эффективность алгоритма, поделив общюю длительность на максимум из отрезков субтитров или речи из алгоритма.

\bigskip

Полный результат работы находится в приложении к данному отчету в файле $output.txt$.

Конечный результат получился следующим:

\bigskip

$Subtitles: 260.061 \; seconds$

$Speech: 233.777 \; seconds$

$Union: 214.228 \; seconds$

$Match: 80.11 \; \%$

\bigskip

В качестве некого бонуса написал скрипт $run$, чтобы не делать постоянно сборку и чистку рабочей директории вручную. Скорее всего, он пока работает только в $Mac OS X$ (в нем есть стандартный $curl$ для скачивания файлов, например), к тому же он больше настроен под меня, так как пока работает нормально только в $z-shell$, но это в общем-то поправимо. В принципе там более-менее понятно, как все это запускать (есть некое подобие $user$-$friendly$ интерфейса). На этом вроде все.

\end{document}